\subsection{Servidor NFS}
Primeramente, recoge las líneas del fichero de configuración, las cuales definen las rutas de los distintos directorios a exportar.

Después, instala el software necesario en la máquina:
\begin{itemize}
    \item Destino local
    \begin{itemize}
        \item nfs-kernel-server
    \end{itemize}
    \item Destino remoto
    \begin{itemize}
        \item sshpass en la máquina local
        \item nfs-kernel-server en la máquina remota
    \end{itemize}
\end{itemize}

Una vez instalado, modificará la variable \texttt{Domain} con el nombre de dominio de la máquina en el fichero \texttt{/etc/idmapd.conf}; y después, para cada uno de los directorios especificados en el fichero de configuración, añadirá los siguientes datos, en líneas por cada directorio a exportar\cite{nfsserver}:

\begin{itemize}
    \item Ruta del directorio
    \item Dirección IP de la red en la que se comparte, con máscara /24
    \item Las opciones para el mismo, siendo las siguientes:
    \begin{itemize}
        \item rw: marca permiso de lectura y escritura
        \item sync: habilita la sincronización del directorio
        \item fsid=0: asigna el id 0 al nuevo Sistema de Ficheros
        \item no\_root\_squash: habilita permisos de root
        \item no\_subtree\_check: deshabilita la comprobación del subárbol de directorios
    \end{itemize}
\end{itemize}

Por último, reiniciará el servicio para cargar la nueva configuración.

\subsection{Cliente NFS}

Primero obtiene las líneas del fichero de configuración, el cual contiene, por cada línea:
\begin{itemize}
    \item La dirección IP del servidor NFS
    \item La ruta del directorio remoto
    \item El punto de montaje del nuevo directorio   
\end{itemize}

Después, instala el software necesario:
\begin{itemize}
    \item Destino local:
    \begin{itemize}
        \item nfs-common
        \item mount
    \end{itemize}
    \item Destino remoto:
    \begin{itemize}
        \item sshpass en la máquina local
        \item nfs-common en la máquina remota
        \item mount en la máquina remota
    \end{itemize}
\end{itemize}

Para la configuración, primero modifica la variable \texttt{Domain} con el nombre de dominio de la máquina, en el fichero \texttt{/etc/idmapd.conf}, tras lo cual, reinicia el servicio \texttt{nfs-common} con la nueva configuración.

Luego, para cada línea del fichero de configuración del servicio, monta el directorio por nfs desde la ruta en la máquina de origen a la ruta en la máquina de destino. 

Finalmente, añade la relación entre cada pareja de directorios al fichero \texttt{/etc/fstab}, de forma que se monte cada vez que se inicie sesión.