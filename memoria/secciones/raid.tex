Esta función creará un sistema RAID en la máquina de destino especificada.

Las líneas del fichero de configuración del servicio se definen como sigue:
\begin{enumerate}
    \item Nombre del dispositivo RAID
    \item Nivel/Tipo de RAID
    \item Dispositivos que lo componen
\end{enumerate}

Después de obtener las líneas pertinentes del fichero de configuración, comprueba e instala los paquetes necesarios usando el método auxiliar \texttt{check\_and\_install\_software}:
\begin{itemize}
    \item Localmente:
    \begin{itemize}
        \item mdadm 
    \end{itemize}
    \item En remoto:
    \begin{itemize}
        \item sshpass localmente
        \item mdadm en la máquina remota
    \end{itemize}
\end{itemize}

Finalmente, usará el comando \texttt{mdadm}\cite{raid} en la máquina de destino (o mediante el comando \texttt{sshpass} en local) para crear el dispositivo RAID con el nombre, nivel y dispositivos especificados en el fichero de configuración.