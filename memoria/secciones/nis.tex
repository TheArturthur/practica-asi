Esta función, al igual que las siguientes, se compone de una configuración distinta para máquinas servidor y cliente.

\subsection{Servidor NIS}
Después de obtener la línea del fichero de configuración, la cual contiene el nombre del que será el servidor NIS, instala los siguientes paquetes software:
\begin{itemize}
    \item Destino local:
    \begin{itemize}
        \item nis
    \end{itemize}
    \item Destino remoto:
    \begin{itemize}
        \item sshpass localmente
        \item nis en la máquina remota
    \end{itemize}
\end{itemize}
\small{\textbf{NOTA:} Al realizar la instalación, es requisito que el usuario introduzca el nombre del servidor NIS.}\\

Después de esto, configura los siguientes ficheros creados por el programa de instalación\cite{nisserver}:
\begin{itemize}
    \item \texttt{/etc/default/nis}: Al que modificará la variable \textbf{NISSERVER}, dándole el valor \textit{master}.
    \item \texttt{/etc/ypserv.securenets}: Comentará la dirección IP por defecto (0.0.0.0) y añadirá la dirección IP de la red a la que está conectada la máquina con máscara /24 (255.255.255.0).
    \item \texttt{/etc/default/nis}: Modificará también la variable \textbf{MERGEGROUP}, de \textit{FALSE} a \textit{TRUE}.
    \item \texttt{/etc/hosts}: Añadirá la dirección IP de red, junto al nombre del host y el nombre del servidor NIS.
\end{itemize}

Para terminar, aplica la configuración con el mandato \texttt{/usr/lib/yp/ypinit -m} y reinicia el servicio NIS creado.

\subsection{Cliente NIS}
