Esta función se encarga de montar el Sistema de Ficheros dado en el directorio de destino especificado, creándolo en caso de que no existiera.

Para ello, lo primero que hace es usar el método auxiliar \texttt{get\_config\_lines} para obtener las líneas del fichero de configuración, en las que aparece el nombre del sistema de ficheros a montar y del directorio de destino de montaje. Las líneas deberán ser de la siguiente manera:
\begin{enumerate}
    \item Nombre del dispositivo
    \item Directorio de montaje
\end{enumerate}

Luego, hace uso de la función \texttt{check\_and\_install\_software} para instalar, sea local o remotamente, los paquetes necesarios para realizar el montaje. Estos paquetes son:
\begin{itemize}
    \item \textbf{Montaje local:}
    \begin{itemize}
        \item mount
    \end{itemize}
    \item \textbf{Montaje remoto:}
    \begin{itemize}
        \item sshpass localmente
        \item mount en el equipo de destino
    \end{itemize}
\end{itemize}

Finalmente, crea el directorio de destino si no existía y luego monta el sistema de ficheros en él, haciendo uso del comando \texttt{mount}\cite{mount}.