Este método crea un Volumen Lógico en la máquina destino especificada.

Comprobará, a partir de la obtención de las líneas de configuración, que haya al menos tres líneas, puesto que el volumen lógico queda definido de la siguiente forma:
\begin{enumerate}
    \item El nombre del volumen
    \item Lista de dispositivos que lo componen
    \item Nombre del volumen 1 y tamaño del mismo
    \item Nombre del volumen 2 y tamaño del mismo
    \item Nombre del volumen 3 y tamaño del mismo
    \item etc.
\end{enumerate}
En caso de no haber al menos tres líneas en el fichero de configuracíon, devolverá error.

Seguidamente, al igual que en los servicios anteriores, instalará el software necesario en las máquinas:
\begin{itemize}
    \item Destino local:
    \begin{itemize}
        \item pvcreate 
        \item vgcreate
        \item lvcreate
    \end{itemize}
    \item Destino remoto:
    \begin{itemize}
        \item sshpass localmente
        \item pvcreate en la máquina remota
        \item vgcreate en la máquina remota
        \item lvcreate en la máquina remota
    \end{itemize}
\end{itemize}

También obtendrá, a partir de la tercera línea del fichero de configuración (y de las siguientes, si procediera), una lista con todos los nombres de los dispositivos y otra con todos los tamaños. Ambas listas están ordenadas de forma que compartan el índice, según la función auxiliar \texttt{get\_lvm\_names\_and\_sizes} definida anteriormente.

Finalmente, seguirá la siguiente secuencia:
\begin{enumerate}
    \item Proceder a crear un grupo con la lista de dispositivos, con el comando \texttt{pvcreate}\cite{lvm}. 
    \item Crear el volumen lógico asociado a ese grupo, con el nombre especificado, usando el comando \texttt{vgcreate}\cite{lvm}.
    \item Añadir los dispositivos y tamaños especificados, en orden, con el comando \texttt{lvcreate}\cite{lvm}.
\end{enumerate}